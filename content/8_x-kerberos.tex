\section{Kerberos}
Kerberos uses symmetric key cryptographic technology.
The master key data is not transferred on network directly
In practice, the three security components in the Kerberos protocol are represented as:
\begin{enumerate}
  \item A client seeking authentication
  \item A server the client wants to access
  \item The ticketing service or key distribution center (KDC)
\end{enumerate}


\subsection{Kerberos Authentication}
\begin{enumerate}
  \item The user shares their username, password, and domain name with the client.
  \item The client assembles a package - or an authenticator - which contains all relevant information about the client, including the user name, date and time. All information contained in the authenticator, aside from the user name, is encrypted with the users password.
  \item The client sends the encrypted authenticator to the KDC.
  \item The KDC checks the user name to establish the identity of the client. The KDC then checks the AD database for the users password. It then attempts to decrypt the authenticator with the password. If the KDC is able to decrypt the authenticator, the identity of the client is verified.
  \item Once the identity of the client is verified, the KDC creates a ticket or session key, which is also encrypted and sent to the client.
  \item The ticket or session key is stored in the clients Kerberos tray; the ticket can be used to access the server for a set time period, which is typically 8 hours.
  \item If the client needs to access another server, it sends the original ticket to the KDC along with a request to access the new resource.
  \item The KDC decrypts the ticket with its key. (The client does not need to authenticate the user because the KDC can use the ticket to verify that the users identity has been confirmed previously).
  \item The KDC generates an updated ticket or session key for the client to access the new shared resource. This ticket is also encrypted by the servers key. The KDC then sends this ticket to the client.
  \item The client saves this new session key in its Kerberos tray, and sends a copy to the server.
  \item The server uses its own password to decrypt the ticket.
  \item If the server successfully decrypts the session key, then the ticket is legitimate. The server will then open the ticket and review the access control list (ACL) to determine if the client has the necessary permission to access the resource.
\end{enumerate}

