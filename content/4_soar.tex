\section{SOAR}
Heisst: \textbf{Security Orchestration, Automation and Response Solutions}\\

SOAR sammelt ebenfalls Daten aus verschiedenen Quellen, ähnlich wie ein SIEM, aber SOAR unterstützt den Incident Responder bei der Bewältigung der Krise.
SOAR ermöglicht ein automatisiertes Eingreifen, wenn ein Sicherheitsvorfall eintritt.
Ein SOAR-System unterstützt den Incident-Responder auch bei der Einführung von Sicherheitsmaßnahmen (Active Directory).
\begin{itemize}
  \item Untersuchung von Alarmen
  \item Orchestrierung (automatisierte Konfiguration, Verwaltung und Koordinierung)
  \item Automatisierung von Arbeitsabläufen
\end{itemize}

\subsection{SOAR vs. SIEM}
Ein \textbf{SOAR} ermöglicht es dem Sicherheitsteam, die Alerts schnell und effizient zu bewältigen, so dass Zeit für wichtige, fachspezifische Aufgaben bleibt, was zu einem leistungsfähigeren SOC führt.
Ein \textbf{SIEM} sammelt vorallem Daten und generiert Alerts.

\subsection{Velociraptor}
