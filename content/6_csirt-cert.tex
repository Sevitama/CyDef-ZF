\section{CERT \& CSIRT}
Heisst: \textbf{Computer Emergency Response Team}\\
Trademark, not so precise.\\
Heisst: \textbf{Computer Emergency Security Incident Response Team}\\
More precise, free to use.

\subsection{Was machen CSIRTs}
Ein CSIRT ist eine Gruppe, die auf Securityincidents reagiert, wenn diese auftreten. 
Zu den wichtigsten Aufgaben eines CSIRT gehören:
\begin{itemize}
  \item Erstellung und Pflege eines Incident Response Plans (IRP)
  \item Untersuchung und Analyse von Incidents
  \item Verwaltung der internen Kommunikation und Aktualisierung während oder unmittelbar nach Incidents
  \item Kommunikation mit Mitarbeitern, Aktionären, Kunden und der Presse über Incidents nach Bedarf
  \item Behebung von Incidents
  \item Empfehlung von Änderungen in den Bereichen Technologie, Richtlinien, Unternehmensführung und Schulung nach Securityincidents
\end{itemize}

Insgesamt analysiert ein CSIRT die Daten von Incidents, diskutiert Beobachtungen und gibt Informationen im gesamten Unternehmen weiter.